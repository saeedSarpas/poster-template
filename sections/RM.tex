\begin{section}{RM}{Dark Galaxy Candidates at Redshift ~3.5 Detected with
    MUSE}{(Dr. Raffaela Marino, Post-Doc at D-PHYS - in collaboration)}
  \begin{minipage}[l]{\textwidth}

    {\small Recent theoretical models suggest that the early phase of galaxy
      formation involves an epoch when galaxies are gas-rich but inefficient
      at forming stars: a ``dark galaxy'' phase. We perform an integral field
      survey for dark galaxies fluorescently illuminated by quasars at z>3
      with MUSE, which provides us a nearly uniform sensitvity coverage over a
      large volume in redshift space, compared to previous narrow-band imaging
      surveys. By comparing the rest-frame equivalent width (EW\_0)
      distributions of the Ly$\alpha$ sources detected in proximity to the
      quasars and in control samples, we detect a clear correlation between
      the locations of high EW\_0 objects and the quasars, not seen in other
      properties such as Ly$\alpha$ luminosities or volume overdensities,
      suggesting the possible fluorescent nature of at least some of these
      objects. Among these, we found 6 dark galaxy candidates with EW\_0 limits
      larger than 240 Angstrom with similar properties to previously detected
      candidates at z~2.4. Our results also provide a lower limit of 60 Myr on
      the quasar lifetime.}
  \end{minipage}

  \vspace{0.5cm}

  \begin{minipage}[t]{\linewidth}
    \begin{center}
      \includegraphics[height=11cm]{RM/EW_LUM_highz.png}
      \includegraphics[height=10cm]{RM/dark-galaxies.png}
    \end{center}
  \end{minipage}

  \vspace{0.5cm}

  {\footnotesize \textit{[Marino et al. 2017, arXiv:1709.03522]}}
\end{section}